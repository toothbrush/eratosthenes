\documentclass[a4paper]{article}

\usepackage{lgrind}
\usepackage{graphicx}
\usepackage{hyperref}

\author{Paul van der Walt\\\url{paul@denknerd.org}}
\date{\today}
\title{Parallel Algorithms: Eratosthenes' Sieve}

\begin{document}
\maketitle
\begin{abstract}
    In this report the findings are presented after benchmarking the Dutch
    supercomputer Huygens and a personal computer. The Sieve of Eratosthenes is
    then implemented in sequence and parallel (including a number of performance
    improvements), and tested on Huygens and a
    personal computer. 
\end{abstract}
\tableofcontents

\section{Introduction}

The so-called sieve of Eratosthenes is an old and unsophisticated method for
finding all primes up to a given number $N$, which lends itself quite nicely to
parallelisation. The idea is roughly as follows: start with the lowest known
prime $i$ (given: $i \leftarrow 2$ in the first iteration) and cross all
multiples of $i$ out of your list of potential primes, which starts off with all
natural numbers 1\ldots$N$. While $i$ is less than $N$, repeat. The next lowest
known prime is the smallest number which hasn't been crossed off the list yet.
More information about Eratosthenes' prime sieve can be found in 

\section{Benchmarks}

\section{Prime Sieve}

\appendix
\section{Sequential sieve code}

\lgrindfile{../seq/sieve.lg}

\section{Parallel sieve code}

\lgrindfile{../par/bspsieve.lg}


\begin{thebibliography}{99}
    \bibitem[SPIN]{SPIN} \url{http://www.spinroot.com}, homepage of the SPIN
        automatic model checker. 
\end{thebibliography}

\end{document}
