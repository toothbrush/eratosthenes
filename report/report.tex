\documentclass[a4paper]{article}

\usepackage{lgrind}
\usepackage{graphicx}
\usepackage{hyperref}

\author{Paul van der Walt\\\url{paul@denknerd.org}}
\date{\today}
\title{Parallel Algorithms: Eratosthenes' Sieve}

\begin{document}
\maketitle
\begin{abstract}
    In this report the findings are presented after benchmarking the Dutch
    supercomputer Huygens and a personal computer. The Sieve of Eratosthenes is
    then implemented in sequence and parallel (including a number of performance
    improvements), and tested on Huygens and a
    personal computer. 
\end{abstract}
\tableofcontents

\section{Introduction}

This report documents the use of BSP-style\cite{BSP} parallel programming to
find primes in parallel, using Eratosthenes' method. 

The so-called sieve of Eratosthenes is an old and unsophisticated method for
finding all primes up to a given number $N$, which lends itself quite nicely to
parallelisation. The idea is roughly as follows: start with the lowest known
prime $i$ (given: $i \leftarrow 2$ in the first iteration) and cross all
multiples of $i$ out of your list of potential primes, which starts off with all
natural numbers 1\ldots$N$. While $i$ is less than $N$, repeat. The next lowest
known prime is the smallest number which hasn't been crossed off the list yet.
More information about Eratosthenes' prime sieve can be found in ???. 

Secondly, we take a look at benchmarking BSP computers and the performance
measured when testing on Huygens\cite{sarahuygens}, the Dutch national
supercomputer, and on a recent MacBook. Specifically, the benchmark is aimed at
parallel computers, measuring not only computation speed, but also
synchronisation time and communication speed. 



\section{Prime Sieve}

\section{Benchmarks}

\appendix
\section{Sequential sieve code}

\lgrindfile{../seq/sieve.lg}

\section{Parallel sieve code}

\lgrindfile{../par/bspsieve.lg}


\begin{thebibliography}{99}
    \bibitem[BSP]{BSP} \url{http://www.bsp-worldwide.org}, homepage of the BSP
        association. 
    \bibitem[SHuy]{sarahuygens}
        \url{https://subtrac.sara.nl/userdoc/wiki/huygens/description},
        information page on the Huygens supercomputer. 
\end{thebibliography}

\end{document}
